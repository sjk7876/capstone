Video analysis has become an integral part of modern sports training, offering athletes and coaches valuable feedback on performance. However, existing tools for volleyball are often limited in precision, accessibility, or scope. In particular, serve analysis remains a largely manual process, relying on subjective visual review rather than quantitative, automated measurements. This project explores how computer vision can be leveraged to develop a lightweight, single-camera system for automated serve analysis, directly from raw video.

\subsection{Volleyball Overview}
Volleyball is a fast-paced team sport that relies on precision, timing, coordination, and teamwork. Each rally starts with a serve, followed by a sequence of passes, sets, and attacks that aim to ground the ball on the opponent's court. A standard indoor court measures 18 by 9 meters, with a net that is 2.43 meters high in men's play and 2.24 meters high in women's play. Teams rotate through six positions, with each player alternating between serving, receiving, and front-row attacking roles throughout a match.

Despite its simple appearance, volleyball is defined by highly technical movements. Serves, spikes, and passes all depend on millisecond-level timing and consistent mechanics. This makes volleyball an interesting challenge for computer vision: tracking a small, fast-moving ball through occlusions, varied lighting, and motion blur while also understanding spatial context on the court. The presence of players moving across the frame, partially obscuring the ball, adds further complexity to automated analysis.

\subsection{Importance of the Serve}
The serve is the only action in volleyball that is completely under one player's control, making it one of the most critical components of the game. Each player on the court is guaranteed to serve at least once per set, and serves can directly score points if they land on the opponent's court. A strong serve can pressure the opponent's receive formation, while a weak or inconsistent one often results in free points for the opponent. Because every rally begins with a serve, it often dictates early momentum and sets the tone for subsequent plays.

Serving is both a mechanical and strategic skill. A consistent serve applies pressure by forcing predictable passes, thereby limiting the opponent's offensive options. Advanced servers vary spin, velocity, and placement to disrupt rhythm or target specific receivers. Different serve types, such as float, topspin, and jump serves, produce distinct flight paths based on their spin characteristics. A float serve, struck with minimal rotation, moves erratically through the air due to turbulent airflow, making it harder to predict. In contrast, a topspin serve generates a sharp downward curve, trading unpredictability for control and speed. Mastering both types requires precise timing of the toss, controlled contact, and consistent follow-through.

Consistency and accuracy are essential: even small deviations in toss height, contact angle, or foot position can cause the ball to drift or miss the target zone. Each serve can move at speeds up to 80 miles per hour, landing on the opponent's court in less than one second. Because the window for error is so small, even experienced players rely heavily on video feedback to refine their technique. However, traditional video review has limitations, providing only qualitative feedback.

\subsection{Motivation for Automated Analysis}
While video review is standard in volleyball training, it is still mostly done manually: slow, subjective, and inconsistent. Athletes often pause and replay clips frame by frame to estimate toss height, landing location, or flight path, but there is no consistent quantitative feedback. This makes it difficult to measure improvement or compare serves between players. Even with slow-motion playback, estimating exact ball paths or landing zones is imprecise and highly dependent on the observer's perception.

Some modern tools, such as BallTimeAI, have started automating volleyball video analysis using proprietary AI models. However, these systems are typically closed-source, subscription-based, and optimized for match-level statistics rather than detailed motion or serve-level analysis. They often focus on game outcomes, such as rally length or attack success rate, rather than the individual skills. Moreover, their reliance on specialized equipment or multiple camera angles limits accessibility for athletes and coaches who typically record sessions with a single baseline camera.

Computer vision offers a way to bridge the gap. With reliable ball tracking and court calibration, it becomes possible to automatically extract serve metrics such as toss consistency, trajectory shape, and landing accuracy straight from normal video. The goal is not to replace coaches, but to make feedback faster, objective, and repeatable. 

Developing such a system also presents technical challenges. The ball's small size, fast speed, and motion blur make detection difficult, especially with a single viewpoint. Background clutter, changing lighting, and occlusion from players crossing the frame introduce further noise. Achieving accurate tracking and trajectory estimation under these conditions requires models capable of understanding both temporal motion and spatial structure. Overcoming these challenges is central to this project's motivation.

\subsection{Project Scope and Objectives}
This project focuses on building a single-camera system for automatic serve analysis using computer vision. The goal is to detect, track, and evaluate volleyball serves from standard baseline footage without any specialized hardware. By combining modern detection, segmentation, and calibration techniques, the system aims to extract meaningful performance metrics directly from video.

The scope includes five main objectives:
\begin{itemize}
    \item \textbf{Ball Detection and Tracking:} Train and evaluate a YOLOv8-based model to detect and track the volleyball across frames during the serve.
    \item \textbf{Court Line Segmentation:} Use a U-Net architecture to segment court lines and generate a homography map for pixel-to-meter calibration.
    \item \textbf{Serve Event Segmentation:} Automatically isolate individual serve sequences (toss, flight, landing) from raw session footage.
    \item \textbf{Trajectory and Consistency Analysis:} Predict and visualize the ball's 2D trajectory, landing zone, and serve consistency across multiple players and sessions.
    \item \textbf{Visualization and Reporting:} Create a user-friendly interface to display serve metrics and visualizations for coaches and athletes.
\end{itemize}

Together, these components form a baseline system for automated serve evaluation: providing objective, repeatable feedback that can support coaching, training, and future sports analytics research. All models will be trained and tested on a custom dataset of 50-100 serves collected across multiple players and recording sessions. Beyond this initial implementation, the system can serve as a foundation for broader extensions, such as 3D trajectory reconstruction, automatic court calibration across different gym environments, and real-time serve feedback applications.